\documentclass[9pt,twocolumn,twoside,]{pnas-new}

%% Some pieces required from the pandoc template
\providecommand{\tightlist}{%
  \setlength{\itemsep}{0pt}\setlength{\parskip}{0pt}}

% Use the lineno option to display guide line numbers if required.
% Note that the use of elements such as single-column equations
% may affect the guide line number alignment.


\usepackage[T1]{fontenc}
\usepackage[utf8]{inputenc}

% For Pandoc highlighting needs
\usepackage{color}
\usepackage{fancyvrb}
\newcommand{\VerbBar}{|}
\newcommand{\VERB}{\Verb[commandchars=\\\{\}]}
\DefineVerbatimEnvironment{Highlighting}{Verbatim}{commandchars=\\\{\}}
% Add ',fontsize=\small' for more characters per line
\newenvironment{Shaded}{}{}
\newcommand{\AlertTok}[1]{\textcolor[rgb]{1.00,0.00,0.00}{\textbf{#1}}}
\newcommand{\AnnotationTok}[1]{\textcolor[rgb]{0.38,0.63,0.69}{\textbf{\textit{#1}}}}
\newcommand{\AttributeTok}[1]{\textcolor[rgb]{0.49,0.56,0.16}{#1}}
\newcommand{\BaseNTok}[1]{\textcolor[rgb]{0.25,0.63,0.44}{#1}}
\newcommand{\BuiltInTok}[1]{#1}
\newcommand{\CharTok}[1]{\textcolor[rgb]{0.25,0.44,0.63}{#1}}
\newcommand{\CommentTok}[1]{\textcolor[rgb]{0.38,0.63,0.69}{\textit{#1}}}
\newcommand{\CommentVarTok}[1]{\textcolor[rgb]{0.38,0.63,0.69}{\textbf{\textit{#1}}}}
\newcommand{\ConstantTok}[1]{\textcolor[rgb]{0.53,0.00,0.00}{#1}}
\newcommand{\ControlFlowTok}[1]{\textcolor[rgb]{0.00,0.44,0.13}{\textbf{#1}}}
\newcommand{\DataTypeTok}[1]{\textcolor[rgb]{0.56,0.13,0.00}{#1}}
\newcommand{\DecValTok}[1]{\textcolor[rgb]{0.25,0.63,0.44}{#1}}
\newcommand{\DocumentationTok}[1]{\textcolor[rgb]{0.73,0.13,0.13}{\textit{#1}}}
\newcommand{\ErrorTok}[1]{\textcolor[rgb]{1.00,0.00,0.00}{\textbf{#1}}}
\newcommand{\ExtensionTok}[1]{#1}
\newcommand{\FloatTok}[1]{\textcolor[rgb]{0.25,0.63,0.44}{#1}}
\newcommand{\FunctionTok}[1]{\textcolor[rgb]{0.02,0.16,0.49}{#1}}
\newcommand{\ImportTok}[1]{#1}
\newcommand{\InformationTok}[1]{\textcolor[rgb]{0.38,0.63,0.69}{\textbf{\textit{#1}}}}
\newcommand{\KeywordTok}[1]{\textcolor[rgb]{0.00,0.44,0.13}{\textbf{#1}}}
\newcommand{\NormalTok}[1]{#1}
\newcommand{\OperatorTok}[1]{\textcolor[rgb]{0.40,0.40,0.40}{#1}}
\newcommand{\OtherTok}[1]{\textcolor[rgb]{0.00,0.44,0.13}{#1}}
\newcommand{\PreprocessorTok}[1]{\textcolor[rgb]{0.74,0.48,0.00}{#1}}
\newcommand{\RegionMarkerTok}[1]{#1}
\newcommand{\SpecialCharTok}[1]{\textcolor[rgb]{0.25,0.44,0.63}{#1}}
\newcommand{\SpecialStringTok}[1]{\textcolor[rgb]{0.73,0.40,0.53}{#1}}
\newcommand{\StringTok}[1]{\textcolor[rgb]{0.25,0.44,0.63}{#1}}
\newcommand{\VariableTok}[1]{\textcolor[rgb]{0.10,0.09,0.49}{#1}}
\newcommand{\VerbatimStringTok}[1]{\textcolor[rgb]{0.25,0.44,0.63}{#1}}
\newcommand{\WarningTok}[1]{\textcolor[rgb]{0.38,0.63,0.69}{\textbf{\textit{#1}}}}

% Pandoc citation processing
\newlength{\csllabelwidth}
\setlength{\csllabelwidth}{3em}
\newlength{\cslhangindent}
\setlength{\cslhangindent}{1.5em}
% for Pandoc 2.8 to 2.10.1
\newenvironment{cslreferences}%
  {}%
  {\par}
% For Pandoc 2.11+
\newenvironment{CSLReferences}[2] % #1 hanging-ident, #2 entry spacing
 {% don't indent paragraphs
  \setlength{\parindent}{0pt}
  % turn on hanging indent if param 1 is 1
  \ifodd #1 \everypar{\setlength{\hangindent}{\cslhangindent}}\ignorespaces\fi
  % set entry spacing
  \ifnum #2 > 0
  \setlength{\parskip}{#2\baselineskip}
  \fi
 }%
 {}
\usepackage{calc} % for calculating minipage widths
\newcommand{\CSLBlock}[1]{#1\hfill\break}
\newcommand{\CSLLeftMargin}[1]{\parbox[t]{\csllabelwidth}{#1}}
\newcommand{\CSLRightInline}[1]{\parbox[t]{\linewidth - \csllabelwidth}{#1}\break}
\newcommand{\CSLIndent}[1]{\hspace{\cslhangindent}#1}


\templatetype{pnasresearcharticle}  % Choose template

\title{Título}

\author[a,1,2]{Carlos Lezama}
\author[a,1,2]{Marco Medina}
\author[a,1,2]{Emiliano Ramírez}
\author[a,1,2]{Santiago Villarreal}

  \affil[a]{Instituto Tecnológico Autónomo de México}


% Please give the surname of the lead author for the running footer
\leadauthor{Lezama, Medina, Ramírez y Villarreal}

% Please add here a significance statement to explain the relevance of your work
\significancestatement{}


\authorcontributions{}


\equalauthors{\textsuperscript{1} Todos los autores contribuyeron a este
trabajo por igual.}

\correspondingauthor{\textsuperscript{2} Trabajo presentado para el
curso de \textbf{Simulación (EST-24107)} impartido por Jorge Francisco
de la Vega Góngora. E-mail:
\href{mailto:jorge.delavegagongora@gmail.com}{\nolinkurl{jorge.delavegagongora@gmail.com}}}

% Keywords are not mandatory, but authors are strongly encouraged to provide them. If provided, please include two to five keywords, separated by the pipe symbol, e.g:
 \keywords{  análisis bayesiano |  aproximación
estocástica |  estimación |  muestreo de importancia  } 

\begin{abstract}
Please provide an abstract of no more than 250 words in a single
paragraph. Abstracts should explain to the general reader the major
contributions of the article. References in the abstract must be cited
in full within the abstract itself and cited in the text.
\end{abstract}

\dates{This manuscript was compiled on \today}
\doi{\url{www.pnas.org/cgi/doi/10.1073/pnas.XXXXXXXXXX}}

\begin{document}

% Optional adjustment to line up main text (after abstract) of first page with line numbers, when using both lineno and twocolumn options.
% You should only change this length when you've finalised the article contents.
\verticaladjustment{-2pt}

\maketitle
\thispagestyle{firststyle}
\ifthenelse{\boolean{shortarticle}}{\ifthenelse{\boolean{singlecolumn}}{\abscontentformatted}{\abscontent}}{}

% If your first paragraph (i.e. with the \dropcap) contains a list environment (quote, quotation, theorem, definition, enumerate, itemize...), the line after the list may have some extra indentation. If this is the case, add \parshape=0 to the end of the list environment.

\acknow{}

\hypertarget{introducciuxf3n}{%
\section*{Introducción}\label{introducciuxf3n}}
\addcontentsline{toc}{section}{Introducción}

Ejemplo de cita (1).

\lipsum[2-4]

\hypertarget{datos}{%
\section*{Datos}\label{datos}}
\addcontentsline{toc}{section}{Datos}

\lipsum[4-6]

\hypertarget{muxe9todos}{%
\section*{Métodos}\label{muxe9todos}}
\addcontentsline{toc}{section}{Métodos}

\lipsum[6-8]

\hypertarget{resultados}{%
\section*{Resultados}\label{resultados}}
\addcontentsline{toc}{section}{Resultados}

\begin{Shaded}
\begin{Highlighting}[]
\NormalTok{gamma.moments }\OtherTok{\textless{}{-}} \ControlFlowTok{function}\NormalTok{(}
\NormalTok{        data, iters, alpha}\FloatTok{.0}\NormalTok{, beta}\FloatTok{.0}
\NormalTok{) \{}
\NormalTok{  sample.mean }\OtherTok{\textless{}{-}} \FunctionTok{mean}\NormalTok{(data)}
\NormalTok{  sample.var }\OtherTok{\textless{}{-}} \FunctionTok{var}\NormalTok{(data)}
\NormalTok{  mu}\FloatTok{.0} \OtherTok{\textless{}{-}} \FunctionTok{c}\NormalTok{(sample.mean, sample.var)}
\NormalTok{  theta }\OtherTok{\textless{}{-}} \FunctionTok{matrix}\NormalTok{(}\ConstantTok{NA}\NormalTok{, }\DecValTok{2}\NormalTok{, iters)}
\NormalTok{  theta[, }\DecValTok{1}\NormalTok{] }\OtherTok{\textless{}{-}} \FunctionTok{c}\NormalTok{(alpha}\FloatTok{.0}\NormalTok{, beta}\FloatTok{.0}\NormalTok{)}

  \ControlFlowTok{for}\NormalTok{ (i }\ControlFlowTok{in} \DecValTok{2}\SpecialCharTok{:}\NormalTok{iters) \{}
\NormalTok{    n }\OtherTok{\textless{}{-}}\NormalTok{ i }\SpecialCharTok{+} \DecValTok{100}
\NormalTok{    simulated }\OtherTok{\textless{}{-}} \FunctionTok{rgamma}\NormalTok{(}
\NormalTok{            n,}
            \AttributeTok{shape =}\NormalTok{ theta[}\DecValTok{1}\NormalTok{, i }\SpecialCharTok{{-}} \DecValTok{1}\NormalTok{],}
            \AttributeTok{scale =}\NormalTok{ theta[}\DecValTok{2}\NormalTok{, i }\SpecialCharTok{{-}} \DecValTok{1}\NormalTok{]}
\NormalTok{    )}
\NormalTok{    mu }\OtherTok{\textless{}{-}} \FunctionTok{c}\NormalTok{(}\FunctionTok{mean}\NormalTok{(simulated), }\FunctionTok{var}\NormalTok{(simulated))}
\NormalTok{    mu.hat }\OtherTok{\textless{}{-}} \FunctionTok{matrix}\NormalTok{(}\DecValTok{0}\NormalTok{, }\DecValTok{2}\NormalTok{, }\DecValTok{2}\NormalTok{)}
\NormalTok{    h }\OtherTok{\textless{}{-}}\NormalTok{ u }\OtherTok{\textless{}{-}} \ConstantTok{NULL}

    \ControlFlowTok{for}\NormalTok{ (j }\ControlFlowTok{in} \DecValTok{1}\SpecialCharTok{:}\FunctionTok{length}\NormalTok{(simulated)) \{}
\NormalTok{      u[}\DecValTok{1}\NormalTok{] }\OtherTok{\textless{}{-}} \SpecialCharTok{{-}}\FunctionTok{digamma}\NormalTok{(theta[}\DecValTok{1}\NormalTok{, i }\SpecialCharTok{{-}} \DecValTok{1}\NormalTok{]) }\SpecialCharTok{{-}}
              \FunctionTok{log}\NormalTok{(theta[}\DecValTok{2}\NormalTok{, i }\SpecialCharTok{{-}} \DecValTok{1}\NormalTok{]) }\SpecialCharTok{+}
              \FunctionTok{log}\NormalTok{(simulated[j])}
\NormalTok{      u[}\DecValTok{2}\NormalTok{] }\OtherTok{\textless{}{-}}\NormalTok{ (}\SpecialCharTok{{-}}\NormalTok{theta[}\DecValTok{1}\NormalTok{, i }\SpecialCharTok{{-}} \DecValTok{1}\NormalTok{] }\SpecialCharTok{/}\NormalTok{ theta[}\DecValTok{2}\NormalTok{, i }\SpecialCharTok{{-}} \DecValTok{1}\NormalTok{]) }\SpecialCharTok{+}
\NormalTok{              simulated[j] }\SpecialCharTok{/}\NormalTok{ (theta[}\DecValTok{2}\NormalTok{, i }\SpecialCharTok{{-}} \DecValTok{1}\NormalTok{]}\SpecialCharTok{\^{}}\DecValTok{2}\NormalTok{)}
\NormalTok{      h[}\DecValTok{1}\NormalTok{] }\OtherTok{\textless{}{-}}\NormalTok{ simulated[j]}
\NormalTok{      h[}\DecValTok{2}\NormalTok{] }\OtherTok{\textless{}{-}}\NormalTok{ (simulated[j] }\SpecialCharTok{{-}} \FunctionTok{mean}\NormalTok{(simulated))}\SpecialCharTok{\^{}}\DecValTok{2}
\NormalTok{      m }\OtherTok{\textless{}{-}}\NormalTok{ h }\SpecialCharTok{\%*\%} \FunctionTok{t}\NormalTok{(u)}
\NormalTok{      mu.hat }\OtherTok{\textless{}{-}}\NormalTok{ mu.hat }\SpecialCharTok{+}\NormalTok{ m}
\NormalTok{    \}}

\NormalTok{    mu.hat }\OtherTok{\textless{}{-}}\NormalTok{ mu.hat }\SpecialCharTok{/} \FunctionTok{length}\NormalTok{(simulated)}

\NormalTok{    par }\OtherTok{\textless{}{-}}\NormalTok{ theta[, i }\SpecialCharTok{{-}} \DecValTok{1}\NormalTok{] }\SpecialCharTok{+}
            \FunctionTok{solve}\NormalTok{(mu.hat) }\SpecialCharTok{\%*\%}\NormalTok{ (mu}\FloatTok{.0} \SpecialCharTok{{-}}\NormalTok{ mu)}

    \ControlFlowTok{if}\NormalTok{ (par[}\DecValTok{1}\NormalTok{] }\SpecialCharTok{*}\NormalTok{ par[}\DecValTok{2}\NormalTok{] }\SpecialCharTok{\textgreater{}} \DecValTok{0}\NormalTok{) \{}
\NormalTok{      theta[, i] }\OtherTok{\textless{}{-}}\NormalTok{ par}
\NormalTok{    \} }\ControlFlowTok{else}\NormalTok{ \{}
\NormalTok{      theta[, i] }\OtherTok{\textless{}{-}}\NormalTok{ theta[, i }\SpecialCharTok{{-}} \DecValTok{1}\NormalTok{] }\SpecialCharTok{+}
              \FunctionTok{runif}\NormalTok{(}\DecValTok{1}\NormalTok{)}
\NormalTok{    \}}
\NormalTok{  \}}

\NormalTok{  theta }\OtherTok{\textless{}{-}} \FunctionTok{data.frame}\NormalTok{(}
          \AttributeTok{x =}\NormalTok{ theta[}\DecValTok{1}\NormalTok{,],}
          \AttributeTok{y =}\NormalTok{ theta[}\DecValTok{2}\NormalTok{,],}
          \AttributeTok{n =} \DecValTok{1}\SpecialCharTok{:}\NormalTok{iters}
\NormalTok{  )}

\NormalTok{  p}\FloatTok{.1} \OtherTok{\textless{}{-}} \FunctionTok{ggplot}\NormalTok{(theta) }\SpecialCharTok{+}
          \FunctionTok{geom\_line}\NormalTok{(}\FunctionTok{aes}\NormalTok{(}\AttributeTok{x =}\NormalTok{ n, }\AttributeTok{y =}\NormalTok{ x),}
                    \AttributeTok{size =} \FloatTok{0.1}\NormalTok{) }\SpecialCharTok{+}
          \FunctionTok{labs}\NormalTok{(}\AttributeTok{title =} \ConstantTok{NULL}\NormalTok{,}
               \AttributeTok{x =} \StringTok{"n"}\NormalTok{,}
               \AttributeTok{y =} \FunctionTok{expression}\NormalTok{(alpha))}

\NormalTok{  p}\FloatTok{.2} \OtherTok{\textless{}{-}} \FunctionTok{ggplot}\NormalTok{(theta) }\SpecialCharTok{+}
          \FunctionTok{geom\_line}\NormalTok{(}\FunctionTok{aes}\NormalTok{(}\AttributeTok{x =}\NormalTok{ n, }\AttributeTok{y =}\NormalTok{ y),}
                    \AttributeTok{size =} \FloatTok{0.1}\NormalTok{) }\SpecialCharTok{+}
          \FunctionTok{labs}\NormalTok{(}\AttributeTok{title =} \ConstantTok{NULL}\NormalTok{,}
               \AttributeTok{x =} \StringTok{"n"}\NormalTok{,}
               \AttributeTok{y =} \FunctionTok{expression}\NormalTok{(beta))}

\NormalTok{  shape }\OtherTok{\textless{}{-}} \FunctionTok{mean}\NormalTok{(theta}\SpecialCharTok{$}\NormalTok{x)}
\NormalTok{  scale }\OtherTok{\textless{}{-}} \FunctionTok{mean}\NormalTok{(theta}\SpecialCharTok{$}\NormalTok{y)}

\NormalTok{  dist.mean }\OtherTok{\textless{}{-}} \FunctionTok{mean}\NormalTok{(shape }\SpecialCharTok{*}\NormalTok{ scale)}
\NormalTok{  dist.var }\OtherTok{\textless{}{-}} \FunctionTok{mean}\NormalTok{(shape }\SpecialCharTok{*}\NormalTok{ (scale}\SpecialCharTok{\^{}}\DecValTok{2}\NormalTok{))}

\NormalTok{  test }\OtherTok{\textless{}{-}} \FunctionTok{ks.test}\NormalTok{(}
\NormalTok{          data, }\StringTok{"pgamma"}\NormalTok{,}
          \AttributeTok{shape =}\NormalTok{ shape, }\AttributeTok{scale =}\NormalTok{ scale}
\NormalTok{  )}

\NormalTok{  results }\OtherTok{\textless{}{-}} \FunctionTok{list}\NormalTok{(}
\NormalTok{          test,}
\NormalTok{          p}\FloatTok{.1}\NormalTok{, p}\FloatTok{.2}\NormalTok{,}
\NormalTok{          shape, scale,}
\NormalTok{          dist.mean, dist.var,}
\NormalTok{          sample.mean, sample.var}
\NormalTok{  )}

  \FunctionTok{return}\NormalTok{(results)}
\NormalTok{\}}
\end{Highlighting}
\end{Shaded}

\begin{verbatim}
## 
##  One-sample Kolmogorov-Smirnov test
## 
## data:  data
## D = 0.092, p-value = 0.7
## alternative hypothesis: two-sided
\end{verbatim}

\begin{flushleft}\includegraphics{/home/carlos/Documents/simulation-fall-2021/term-paper/term-paper_files/figure-latex/tests-1} \end{flushleft}

\begin{flushleft}\includegraphics{/home/carlos/Documents/simulation-fall-2021/term-paper/term-paper_files/figure-latex/tests-2} \end{flushleft}

\begin{verbatim}
## [1] 2.499
\end{verbatim}

\begin{verbatim}
## [1] 16.69
\end{verbatim}

\begin{verbatim}
## [1] 41.69
\end{verbatim}

\begin{verbatim}
## [1] 695.8
\end{verbatim}

\begin{verbatim}
## [1] 41.77
\end{verbatim}

\begin{verbatim}
## [1] 691.1
\end{verbatim}

\begin{verbatim}
## 
##  One-sample Kolmogorov-Smirnov test
## 
## data:  data
## D = 0.087, p-value = 0.7
## alternative hypothesis: two-sided
\end{verbatim}

\begin{flushleft}\includegraphics{/home/carlos/Documents/simulation-fall-2021/term-paper/term-paper_files/figure-latex/tests-3} \end{flushleft}

\begin{flushleft}\includegraphics{/home/carlos/Documents/simulation-fall-2021/term-paper/term-paper_files/figure-latex/tests-4} \end{flushleft}

\begin{verbatim}
## [1] 2.476
\end{verbatim}

\begin{verbatim}
## [1] 16.7
\end{verbatim}

\begin{verbatim}
## [1] 41.35
\end{verbatim}

\begin{verbatim}
## [1] 690.4
\end{verbatim}

\begin{verbatim}
## [1] 41.77
\end{verbatim}

\begin{verbatim}
## [1] 691.1
\end{verbatim}

\begin{verbatim}
## 
##  One-sample Kolmogorov-Smirnov test
## 
## data:  data
## D = 0.12, p-value = 0.3
## alternative hypothesis: two-sided
\end{verbatim}

\begin{flushleft}\includegraphics{/home/carlos/Documents/simulation-fall-2021/term-paper/term-paper_files/figure-latex/tests-5} \end{flushleft}

\begin{flushleft}\includegraphics{/home/carlos/Documents/simulation-fall-2021/term-paper/term-paper_files/figure-latex/tests-6} \end{flushleft}

\begin{verbatim}
## [1] 2.62
\end{verbatim}

\begin{verbatim}
## [1] 16.68
\end{verbatim}

\begin{verbatim}
## [1] 43.71
\end{verbatim}

\begin{verbatim}
## [1] 729.1
\end{verbatim}

\begin{verbatim}
## [1] 41.77
\end{verbatim}

\begin{verbatim}
## [1] 691.1
\end{verbatim}

\lipsum[8-10]

\hypertarget{conclusiones}{%
\section*{Conclusiones}\label{conclusiones}}
\addcontentsline{toc}{section}{Conclusiones}

\lipsum[10-12]

\hypertarget{anexos}{%
\section*{Anexos}\label{anexos}}
\addcontentsline{toc}{section}{Anexos}

\hypertarget{referencias}{%
\section*{Referencias}\label{referencias}}
\addcontentsline{toc}{section}{Referencias}

\showmatmethods
\showacknow
\pnasbreak

\hypertarget{refs}{}
\begin{CSLReferences}{0}{0}
\leavevmode\vadjust pre{\hypertarget{ref-mainArt}{}}%
\CSLLeftMargin{1. }
\CSLRightInline{Gelman A (1995)
\href{http://www.jstor.org/stable/1390626}{Method of moments using monte
carlo simulation}. \emph{Journal of Computational and Graphical
Statistics} 4(1):36--54.}

\end{CSLReferences}



% Bibliography
% \bibliography{pnas-sample}

\end{document}
